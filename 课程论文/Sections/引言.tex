\section{越野滑雪}
缆车又颠了一下,停了。没法朝前开了,大雪给风刮得严严实实地积在车道上。冲刷高山裸露表层的狂风把向风一面的雪刮成一层冰壳。尼克正在行李车厢里给滑雪板上蜡,把靴尖塞进滑雪板上的铁夹,牢牢扣上夹子。他从车厢边缘跳下,落脚在硬邦邦的冰壳上,来一个弹跳旋转,蹲下身子,把滑雪杖拖在背后,一溜烟滑下山坡。

乔治在下面的雪坡上一落一起,再一落就不见了人影。尼克顺着陡起陡伏的山坡滑下去时,那股冲势加上猛然下滑的劲儿把他弄得浑然忘却一切,只觉得身子里有一股飞翔、下坠的奇妙感。他挺起身,稍稍来个上滑姿势,一下子又往下滑,往下滑,冲下最后一个陡峭的长坡,越滑越快,越滑越快,雪坡似乎在他脚下消失了。身子下蹲得几乎倒坐在滑雪板上,尽量把重心放低,只见飞雪犹如沙暴,他知道速度太快了。但他稳住了。他决不失手摔倒。随即一搭被风刮进坑里的软雪把他绊倒,滑雪板一阵磕磕绊绊,他接连翻了几个筋斗,觉得活像只挨了枪子的兔子,然后停住,两腿交叉,滑雪板朝天翘起,鼻子和耳朵里满是雪。

乔治站在坡下稍远的地方,正噼噼啪啪地拍掉风衣上的雪。

“你的姿势真美妙,迈克,”他对尼克大声叫道。“那搭烂糟糟的雪真该死。把我也这样绊了一跤。”

“在峡谷滑雪是什么味儿?”尼克仰天躺着,踢蹬着滑雪板,挣扎站起来。

“你得靠左边滑。因为谷底有堵栅栏,所以飞速冲下去后得来个大旋身。”

“等一会儿我们一起去滑。”

“不,你赶快先去。我想看你滑下峡谷。”

尼克·亚当斯赶过背部宽阔、金发上还蒙着一点儿雪的乔治身边向上攀登,他的滑雪板开始有点打滑,随后一下子猛冲下去,把晶莹的雪糁儿擦得嘶嘶响,随着他在起伏不定的峡谷里时上时下,看起来像是在浮上来又沉下去。他坚持靠左边滑,末了,在冲向栅栏时,紧紧并拢双膝,像拧紧螺旋似的旋转身子,把滑雪板向右来个急转弯,扬起滚滚白雪,然后慢慢减速,跟山坡和铁丝栅栏平行地站住了。

他抬头看看山上。乔治正屈起双膝,用特勒马克姿势滑下山来;一条腿在前面弯着,另一条腿在后面拖着,两支滑雪杖像虫子的细腿那样荡着,杖尖触到地面,掀起阵阵白雪,最后,这整个一腿下跪、一腿拖随的身子来个漂亮的右转弯,蹲着滑行,双腿一前一后,飞快移动,身子探出,防止旋转,两支滑雪杖像两个光点,把弧线衬托得更加突出,一切都笼罩在漫天飞舞的白雪中。

“我就怕大旋身,”乔治说,“雪太深了。你做的姿势真美妙。”

“我的一条腿做不来特勒马克,”尼克说。

尼克用滑雪板把铁丝栅栏的最高一股铁丝压下,乔治纵身越过去。尼克跟他来到大路上。他们沿路屈膝滑行,进入一片松林。路面结着光亮的冰层,被拖运原木的马儿拉的犁弄脏了,染得一搭橙红,一搭烟黄。两人一直沿着路边那片雪地滑行。大路陡的往下倾斜通往小河,然后笔直上坡。他们透过林子,看得见一座饱经风吹雨打、屋檐较低的长形的房子。从林子里看,这房子显得泛黄。走近了,看出窗框漆成绿色。油漆在剥落。尼克用一支滑雪杖把滑雪板上的夹靴夹敲松,双脚一踢,让滑雪板掉下。

“我们还是把滑雪板带上去的好,”他说。

他肩起滑雪板,把靴跟的铁钉扎进冰封的立脚点,一步步爬上陡峭的山路。他听见乔治紧跟在后,一边喘息,一边把靴跟扎进冰雪。他们把滑雪板竖靠在客栈的墙上,相互拍掉彼此裤子上的雪,把靴子蹬蹬干净才走进去。

客栈里黑咕隆咚的。有只大瓷火炉在屋角亮着火光。天花板很低。屋内两边那些酒渍斑斑的暗黑色桌子后面摆着光溜溜的长椅。两个瑞士人坐在炉边,一边抽着烟斗,一边喝着小杯浑浊的新酒。尼克和乔治脱去茄克衫,在炉子另一边靠墙坐下。有个人在隔壁房里停止了歌唱,一个围着蓝围裙的姑娘走出门来看看他们想要什么喝的。

“来瓶西昂酒,”尼克说。“行不行,吉奇?”

“行啊,”乔治说。“你对酒比我内行。我什么酒都爱喝。”

姑娘走出去了。

“没一项玩意儿真正比得上滑雪,对吧?”尼克说。“你滑了老长一段路,头一回歇下来时就会有这么个感觉。”

“嘿,”乔治说。“这是妙不可言的。”

姑娘拿酒进来,他们一时拔不出瓶塞。最后还是尼克打开了。姑娘出去了,他们听见她在隔壁房里唱德语歌。

“酒里有些瓶塞渣子没关系,”尼克说。

“不知她有没有糕点。”

“我们问问看。”

姑娘走进屋,尼克注意到她围裙鼓鼓地遮着大肚子。不知她最初进来时我怎么会没看见,他想。

“你唱的什么歌?”他问她。

“歌剧,德国歌剧。”她不愿谈论这个话题。“你们要吃的话,我们有苹果馅卷饼。”

“她不太客气,是不?”乔治说。

“啊,算了。她不认识我们,没准儿当我们要拿她唱歌开玩笑呢。她大概是从北边讲德语的地区来的,待在这里脾气躁,再说,没结婚肚子里就有了这孩子,所以脾气躁,碰不得。”

“你怎么知道她没结婚?”

“没戴戒指。真见鬼,这一带的姑娘都是弄大了肚子才结婚的。”

门开了,一帮子从大路那头来的伐木工人走进来,在屋里把靴子上的雪跺掉,身上直冒水汽。那女招待给这帮人送来了三公升新酒,他们分坐两桌,光抽烟,不作声,脱下了帽,有的背靠着墙,有的趴在桌上。屋外,拉运木雪橇的马儿偶尔一仰脖子,铃铛就清脆地丁当作响。

乔治和尼克都高高兴兴的。他们两人很合得来。他们知道回去还有一段路程可滑呢。

“你几时得回学校去?”尼克问。

“今晚,”乔治回答。“我得赶十点四十分从蒙特勒开出的车。”

“我真希望你能留下过夜,我们明天上百合花峰去滑雪。”

“我得上学啊,”乔治说。“哎呀,尼克,难道你不希望我们能就这么在一起闲逛吗?带上滑雪板,乘上火车,到一个地方滑个痛快,滑好上路,找客栈投宿,再一直越过奥伯兰山脉,直奔瓦莱州,穿过恩加丁谷地,随身背包里只带上修理工具匣和替换毛衣和睡衣,甭管学校啊什么的。”

“对,就这样穿过黑森林区。哎呀,都是好地方啊。”

“就是你今年夏天钓鱼的地方吧?”

“是啊。”

他们吃着苹果馅卷饼,喝干了剩酒。

乔治倒身靠着墙,闭上眼。

“喝了酒我总是这样感觉,”他说。

“感觉不好?”尼克问。

“不。感觉好,只是怪。”

“我明白,”尼克说。

“当然,”乔治说。

“我们再来一瓶好吗?”尼克问。

“我不想喝了,”乔治说。

他们坐在那儿,尼克双肘撑在桌上,乔治往墙上颓然一靠。

“海伦快生孩子了吧?”乔治说,身子离开墙凑到桌上。

“是啊。”

“几时?”

“明年夏末。”

“你高兴吗?”

“是啊。眼前。”

“你打算回美国去吗?”

“看来要回去吧。”

“你想要回去吗?”

“不。”

“海伦呢?”

“不。”

乔治默默坐着。他望着那空酒瓶和那些空酒杯。

“真要命不是?”他说。

“不。还说不上,”尼克说。

“为什么?”

“我不知道,”尼克说。

“你们今后在美国还会一块儿滑雪吗?”乔治说。

“我不知道,”尼克说。

“那些山不怎么样,”乔治说。

“对,”尼克说。“岩石太多。树木也太多,而且都太远。”

“是啊,”乔治说,“加利福尼亚就是这样。”

“是啊,”尼克说,“我到过的地方处处都这样。”

“是啊,”乔治说,“都是这样。”

瑞士人站起身,付了账,走出去了。

“我们是瑞士人就好了,”乔治说。

“他们都有大脖子的毛病,”尼克说。

“我不信,”乔治说。

“我也不信,”尼克说。

两人哈哈大笑。

“也许我们再也没机会滑雪了,尼克,”乔治说。

“我们一定得滑,”尼克说。“要是不能滑就没意思了。”

“我们要去滑,没错,”乔治说。

“我们一定得滑,”尼克附和说。

“希望我们能就此说定了,”乔治说。

尼克站起身。他把风衣扣紧。他朝乔治弯下身子,拿起靠墙放着的两支滑雪杖。他把一支滑雪杖戳在地板上。

“说定了可一点也靠不住,”他说。

他们开了门,走出去。天气很冷。雪结得硬邦邦的。大路一直爬上山坡通到松林里。

他们把刚才靠在客栈墙上的滑雪板拿起来。尼克戴上手套。乔治已经扛着滑雪板上路了。这下子他们可要一起跑回家了。