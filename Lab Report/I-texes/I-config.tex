%% prefer than direct use in usepackage geometry
%% A4 layout in point is % 595x842

%% default value
\setlength{\hoffset}{0pt}
\setlength{\voffset}{0pt}

%% height
%% 72 - 60 + 20 + 25 = 57
\setlength{\topmargin}{-60pt}
\setlength{\headheight}{20pt}
\setlength{\headsep}{25pt}
\setlength{\footskip}{30pt}

%% width
%% 72 + 32 + 10 = 114pt = 40mm
\setlength{\oddsidemargin}{-6pt}
\setlength{\evensidemargin}{-6pt}
\setlength{\marginparsep}{0pt}
\setlength{\marginparwidth}{0pt}

%% size text
\setlength{\textheight}{728pt} % 842 - 60 - 20 - 25 - 30 = 707?
\setlength{\textwidth}{464pt}

%% if you need global ``\noindent''
%\setlength{\parindent}{0pt}

%% style
%% preliminary, just roman pagination + empty header
\fancypagestyle{preliminary}{
    \renewcommand{\headrulewidth}{0pt}
    \fancyhead[RCL]{}

    \pagenumbering{Roman}
}

%% chapter/classic text style
\fancypagestyle{chapter}{
  %% title of the chapter, left header, no uppercase, 10 pt, italics, no bold
  \fancyhead[L]{\normalfont\itshape\fontsize{10pt}{12pt}\selectfont\nouppercase{\leftmark}}
  \fancyhead[R]{Group \# 15}

  \fancyfoot[C]{\thepage}
  \renewcommand{\headrulewidth}{0.4pt}
  \renewcommand{\footrulewidth}{0pt}
  \pagenumbering{arabic}
}

%% define length and scaling for baseline
\newcommand{\headingBaseline}{12}
\newcommand{\headingBaselineDiv}{10}
\newlength{\chapterFontSize}
\newlength{\sectionFontSize}
\newlength{\subsectionFontSize}
\newlength{\chapterBaseline}
\newlength{\sectionBaseline}
\newlength{\subsectionBaseline}

%% change those value if you want to change the chapter/section/subsection font size
\setlength{\chapterFontSize}{14pt}
\setlength{\sectionFontSize}{12pt}
\setlength{\subsectionFontSize}{12pt}

%% automatic computation for baseline, rule of thumb is 1.2
\setlength{\chapterBaseline}{ \chapterFontSize * \headingBaseline / \headingBaselineDiv}
\setlength{\sectionBaseline}{ \sectionFontSize * \headingBaseline / \headingBaselineDiv}
\setlength{\subsectionBaseline}{ \subsectionFontSize * \headingBaseline / \headingBaselineDiv}

%% headings
%% Chapter, 14-point, bold, capitalised initial letter,
\titleformat{\chapter}[display]
    {\normalfont\bfseries\fontsize{\chapterFontSize}{\chapterBaseline}\selectfont}{\chaptertitlename\ \thechapter}{14pt}{\capitalisewords}
%% left|above|below
\titlespacing{\chapter}{0pt}{10pt}{25pt}

%% Section, 12-point
\titleformat{\section}[hang]
    {\normalfont\bfseries\fontsize{\sectionFontSize}{\sectionBaseline}\selectfont}{\thesection}{5pt}{}
%% left|above|below
\titlespacing{\section}{0pt}{25pt}{15pt}

%% Subsection, 12-point, italic
\titleformat{\subsection}[hang]
    {\normalfont\bfseries\itshape\fontsize{\subsectionFontSize}{\subsectionBaseline}\selectfont}{\thesubsection}{5pt}{}
%% left|above|below
\titlespacing{\subsection}{0pt}{20pt}{10pt}

%% table of content
\renewcommand{\contentsname}{Table of Contents}
\setcounter{tocdepth}{2}
\setcounter{secnumdepth}{2}

%% init gloassaries
%% noidx cause otherwise we have to do a normal glossary, compile, then remove it so it is cached
%% because we only use acronym
% \makenoidxglossaries

%% bibliography config
%% https://tex.stackexchange.com/a/6977
\DeclareBibliographyCategory{cited}
\AtEveryCitekey{\addtocategory{cited}{\thefield{entrykey}}}
\addbibresource{Appendices/bibliography.bib}

%% hyperref setup
\hypersetup{
  colorlinks = true,
  linkcolor = blue, % normal internal links, like ref, can be black tbh
  citecolor = blue, % bibliographical links
  urlcolor = blue, % linked urls
  filecolor = black % url which open local files
}

%% modified reference function
%% https://tex.stackexchange.com/a/438998
\newcommand\eref[1]{equation~(\ref{#1})}
\newcommand\tref[1]{table~\ref{#1}}
\newcommand\fref[1]{figure~\ref{#1}}

%% line spacing
\setstretch{1.5}

%% prefer same space width
% \RaggedRight

%% hyphenation
\hyphenpenalty = 100       % 控制断字频率(值越低越积极断字)
\exhyphenpenalty = 50      % 对已包含连字符单词的断字控制
\tolerance = 1000          % 允许更大的行内间距变化
\emergencystretch = 3em    % 允许临时拉伸空间避免溢出

%% enable package ``minted''
\setminted{
    breaklines=true,          % 自动换行
    breaksymbolleft=,         % 移除换行箭头符号(节省空间)
    baselinestretch=0.9,      % 行距压缩到小于1的值
    fontsize=\footnotesize,   % 使用更小的字号
    frame=leftline,           % 仅保留左侧竖线(节省空间)
    framesep=2pt,             % 边框间距压缩到2pt
    rulecolor=gray!30,        % 浅灰色边框(更低调)
    autogobble=true,          % 自动对齐代码缩进
    xleftmargin=5pt,          % 左侧缩进减少
    xrightmargin=0pt,         % 右侧无缩进
    numbersep=3pt,            % 行号间距减少
    linenos=true,             % 关闭行号(节省空间)
    bgcolor=white,            % 白色背景(移除灰色背景节省视觉空间)
}
